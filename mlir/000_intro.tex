% !TeX spellcheck = ru_RU
% !TEX root = vkr.tex

\section*{Введение}
\thispagestyle{withCompileDate}

В современном мире существует множество инструментов, реализованных на разных языках программирования. Для того чтобы при разработке чего-либо на каком-либо языке программирования использовать надёжные готовые решения, написанные на других языках, реализуют привязки (от англ. bindings) --- интерфейс, предоставляющий набор функций для взаимодействия с библиотеками, написанными на других языках программирования.

LLVM \cite{LLVM} --- инфраструктура для разработки компиляторов. LLVM состоит из множества подпроектов. Ядро LLVM базируется на промежуточном представлении LLVM (LLVM IR). LLVM IR это довольно низкоуровневое промежуточное представление, поэтому прямолинейное моделирование на нём высокоуровневых языковых конструкций затруднительно. Чтобы решить эту проблему, а также внедрить специфичные для языка оптимизации добавляют множество уровней абстракции, определяя новые промежуточные представления.

Разработка промежуточных представлений -- хорошо изученная область. Однако стоимость проектирования и внедрения промежуточных представлений всё равно остаётся высокой.

Проект MLIR \cite{MLIR} (Multi-Level Intermediate Representation) --- это подпроект LLVM, направленный на непосредственное решение проблем проектирования и реализации промежуточных представлений за счёт удешевления определения и внедрения новых уровней абстракции и предоставления инфраструктуры для решения распространенных проблем разработки промежуточных представлений.

Проблема в том, что MLIR реализован на языке \Cpp{}, но разработчикам, использующим \OCaml{}, также хотелось бы иметь такой мощный инструмент у себя в арсенале. Таким образом, эта работа посвящена реализации привязок к MLIR для языка программирования \OCaml{}.
