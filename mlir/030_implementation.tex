% !TeX spellcheck = ru_RU
% !TEX root = vkr.tex

\section{Реализация}

\subsection{Привязки}

\subsubsection{Выбор версии MLIR}
MLIR является подпроектом LLVM и входит в монорепозиторий LLVM\footnote{\url{https://github.com/llvm/llvm-project/} (дата доступа:   \DTMdate{2024-01-04}).}, поэтому версия MLIR совпадает с версией LLVM. На момент начала работы версия LLVM 16.0.6 являлась последней версией, к которой существовали привязки\footnote{\url{https://opam.ocaml.org/packages/llvm/llvm.16.0.6+nnp/} (дата доступа:   \DTMdate{2024-01-04}).} для \OCaml{}, поэтому было решено реализовывать привязки к версии MLIR 16.0.6.

\subsubsection{C API}

Из-за применения компиляторами \Cpp{} name mangling и различий в модели памяти создать привязки к \Cpp{} проблематично. Из-за этого обычно привязки делают к \C, который уже вызывает \Cpp{}. Специально для создания привязок MLIR имеет C API\footnote{\url{https://mlir.llvm.org/docs/CAPI/} (дата доступа:   \DTMdate{2024-01-04}).}, предоставляющий API к базовым возможностям MLIR. Поэтому привязки будут строится именно к этому API.

\subsubsection{Первая сборка}

Привязки к версии 16.0.6 строились поверх форка привязок\footnote{\url{https://github.com/tachukao/ocaml-mlir} (дата доступа:   \DTMdate{2024-01-04}).} к форку ML\-IR одиннадцатой версии, поэтому первым делом было необходимо обновить все зависимости и удалить устаревший код. Кроме того, чтобы не тратить время на сборку MLIR из исходников, было решено сделать возможным построение привязок поверх dev пакетов\footnote{\url{https://apt.llvm.org/} (дата доступа:   \DTMdate{2024-01-04}).} MLIR и LLVM.

\subsubsection{Написание привязок}

C API одиннадцатой версии MLIR беднее версии 16.0.6. Поэтому следующим шагом необходимо было реализовать недостающие привязки к C API версии 16.0.6 при помощи библиотеки ctypes \ref{sec:ctypes}.

\subsubsection{Архитектура}

Так как привязки создавались к \C{}, а \C{} относительно \OCaml{} это довольно низкоуровневый язык, поэтому над реализованными привязками находится ещё один уровень абстракции, который позволяет пользователям привязок не задумываться над приведением сложных \OCaml{} типов в примитивные типы \C{}.

Вышеупомянутый дополнительный уровень абстракции над привязками\footnote{\url{https://github.com/s-khechnev/ocaml-mlir/tree/llvm-16.0.6/src/stubs} (дата доступа:   \DTMdate{2024-01-04}).} представляет из себя модуль\footnote{\url{https://github.com/s-khechnev/ocaml-mlir/blob/llvm-16.0.6/src/mlir/mlir.mli} (дата доступа:   \DTMdate{2024-01-04}).}, содержащий мини документацию к привязкам и инкапсулирующий преобразование сложных \OCaml{} типов в необходимые для привязок типы, предоставляемые ctypes, которые представляют типы \C{}. Например, список \OCaml{} преобразуется в пару: указатель на первый элемент и число элементов в списке.

\subsection{Мини язык программирования}

У MLIR есть учебное пособие\footnote{\url{https://mlir.llvm.org/docs/Tutorials/Toy/} (дата доступа:   \DTMdate{2024-01-04}).}, в котором реализуется <<игрушечный>> язык программирования, для быстрого освоения базовых концепций MLIR.

В качестве доказательства работоспособности и примера использования реализованных привязок было решено реализовать вышеупомянутый язык программирования на \OCaml{} с использованием разработанных привязок.

\subsubsection{Описание языка}

Основным и единственным типом данных в данном языке программирования являются тензоры (ранга 0--2, т.~е константы, векторы и матрицы) с элементами типа \texttt{float64}. Язык позволяет производить различные математические операции над тензорами: поэлементное сложение и умножение, транспонирование и изменение формы тензора. Кроме того, можно определять функции и печатать тензоры. Переменные в языке неизменяемы. В языке статическая типизация, есть вывод типов.

\begin{lstlisting}[caption={Пример программы на мини языке программирования.}, frame=single]
def mul_transpose(a, b) {
    return transpose(a) * transpose(b);
}

def main() {
    var a = [[1.1, 2, 3.3], [4, 5.5, 6]];

    # "<2, 3>" reshapes <6> to <2, 3>
    var b<2, 3> = [1, 2.2, 3, 4.4, 5, 6.6];

    var c = mul_transpose(a, b);
    print(c);
}
\end{lstlisting}

\subsubsection*{Процесс компиляции:}

\begin{enumerate}
	\item Парсинг в AST (Abstract Syntax Tree).
	\item \label{mlir_toy} Генерация из AST промежуточного представления, которое состоит из операций диалекта высокого уровня, определённого для данного языка при помощи MLIR.
	\item Вывод типов.
	\item Оптимизация промежуточного представления из шага \ref{mlir_toy}.
	\item \label{affine} Частичное понижение (от англ. lowering) промежуточного представления сгенерированного в шаге \ref{mlir_toy} в промежуточное представление более низкого уровня, состоящее из следующих диалектов:
	      \begin{itemize}
		      \item arith\footnote{\url{https://mlir.llvm.org/docs/Dialects/ArithOps/} (дата доступа:   \DTMdate{2024-01-04}).};
		      \item affine\footnote{\url{https://mlir.llvm.org/docs/Dialects/Affine/} (дата доступа:   \DTMdate{2024-01-04}).};
		      \item func\footnote{\url{https://mlir.llvm.org/docs/Dialects/Func/} (дата доступа:   \DTMdate{2024-01-04}).};
		      \item memref\footnote{\url{https://mlir.llvm.org/docs/Dialects/MemRef/} (дата доступа:   \DTMdate{2024-01-04}).}.
	      \end{itemize}
	\item Оптимизация промежуточного представления из шага \ref{affine}.
	\item \label{llvm_mlir} Понижение промежуточного представления из шага \ref{affine} в промежуточное представление, основанное на диалекте LLVM\footnote{\url{https://mlir.llvm.org/docs/Dialects/LLVM/} (дата доступа:   \DTMdate{2024-01-04}).}.
	\item Понижение промежуточного представления из шага \ref{llvm_mlir} в LLVM IR.
	\item Компиляция LLVM IR при помощи JIT компилятора LLVM.
\end{enumerate}

\subsubsection{AST}

Корнем AST является модуль --- список функций. Функция представляется как пара: прототип --- имя и аргументы, тело. Тело функции состоит из списка выражений.

\begin{lstlisting}[caption={AST языка.}, language=Caml, frame=single]
type shape = int array
type var = string

type expr =
    | Num of float
    | Literal of shape * expr list
    | Var of var
    | VarDecl of string * shape * expr
    | Return of expr option
    | BinOp of [ `Add | `Mul ] * expr * expr
    | Call of string * expr list
    | Print of expr

type proto = Prototype of string * var list
type func = Function of proto * expr list
type modul = func list
\end{lstlisting}

\subsubsection{Парсер}

Парсер для мини языка программирования реализован при помощи парсер-комбинаторов. В качестве библиотеки парсер-комбинаторов использовалась библиотека angstrom\footnote{\url{https://github.com/inhabitedtype/angstrom} (дата доступа:   \DTMdate{2024-01-04}).}, т.к она является самой популярной среди таковых и обладает лучшей производительностью и документацией среди аналогов.

\subsubsection{Диалект для мини языка}

\begin{lstlisting}[caption={Определение диалекта.}, frame=single]
def Toy_Dialect : Dialect {
    let name = "toy";
    let cppNamespace = "::mlir::toy";
}
\end{lstlisting}

\label{toy_ops} Диалект содержит 10 операций:
\begin{enumerate}
	\item toy.constant --- операция, представляющая константу (тензор);
	\item toy.add --- операция сложения тензоров;
	\item toy.mul --- операция умножения тензоров;
	\item toy.cast --- вспомогательная операция для встраивания (inlining) функций;
	\item \label{func} toy.func --- операция, моделирующая функции;
	\item toy.return --- операция возврата значения из функции (операции \ref{func});
	\item toy.generic\_call --- операция, представляющая вызов функции (операции \ref{func});
	\item toy.print --- операция печати тензора в стандартный вывод;
	\item toy.reshape --- операция изменения формы тензора;
	\item toy.transpose --- операция транспонирования тензора.
\end{enumerate}

\begin{lstlisting}[caption={Определение при помощи TableGen операции toy.constant.}, frame=single]
def ConstantOp : Toy_Op<"constant", [Pure]> {
    let summary = "constant";
    let description = [{
        Constant operation turns a literal into an SSA value.
        The data is attached to the operation as an attribute.

        ```mlir
        %0 = toy.constant dense<[[1.0, 2.0], [3.0, 4.0]]>
                                           : tensor<2x2xf64>
        ```
    }];

    // The constant operation takes an attribute as
    // the only input.
    let arguments = (ins F64ElementsAttr);

    // The constant operation returns a single
    // value of TensorType.
    let results = (outs F64Tensor);
    ...
}
\end{lstlisting}

\subsubsection{Генерация из AST промежуточного представления высокого уровня}

Для генерации первого промежуточного представления, содержащего операции из вышеупомянутого диалекта, производился обход AST и последовательная генерация операций диалекта из узлов AST. Создание операции происходит при помощи модуля <<OperationState>>, путём определения имени, позиции в файле, операндов, результатов, и именованных атрибутов будущей операции.

\begin{lstlisting}[caption={Пример программы на мини языке программирования.}, frame=single, label={toy_ex}]
def multiply_transpose(a, b) {
    return transpose(a) * transpose(b);
}

def main() {
    var a<2, 3> = [[1, 2, 3], [4, 5, 6]];
    var b<2, 3> = [1, 2, 3, 4, 5, 6];
    var c = multiply_transpose(a, b);
    print(c);
}
\end{lstlisting}

\begin{lstlisting}[caption={Сгенерированное промежуточное представление высокого уровня для программы листинга \ref{toy_ex}.}, frame=single, label={toy_ir}]
module {
 toy.func @multiply_transpose(%arg0: tensor<*xf64>,
                            %arg1:tensor<*xf64>) -> tensor<*xf64>
 {
    %0 = toy.transpose(%arg0 : tensor<*xf64>) to tensor<*xf64>
    %1 = toy.transpose(%arg1 : tensor<*xf64>) to tensor<*xf64>
    %2 = toy.mul %0, %1 : tensor<*xf64>
    toy.return %2 : tensor<*xf64>
 }

 toy.func @main() {
    %0 = toy.constant dense<[[1.0, 2.0, 3.0], [4.0, 5.0, 6.0]]>
                                            : tensor<2x3xf64>
    %1 = toy.reshape(%0 : tensor<2x3xf64>) to tensor<2x3xf64>
    %2 = toy.constant dense<[1.0, 2.0, 3.0, 4.0, 5.0, 6.0]>
                                            :tensor<6xf64>
    %3 = toy.reshape(%2 : tensor<6xf64>) to tensor<2x3xf64>
    %4 = toy.generic_call @multiply_transpose(%1, %3)
     : (tensor<2x3xf64>, tensor<2x3xf64>) -> tensor<*xf64>
    toy.print %4 : tensor<*xf64>
    toy.return
 }
}
\end{lstlisting}

\subsubsection{Вывод типов (форм тензоров)}

Для дальнейших оптимизаций и генераций более низкоуровневых промежуточных представлений необходимо, чтобы все значения имели тензоры определённой формы (на данный момент это не так. Например, на листинге \ref{toy_ir} переменная \%4 имеет тип \texttt{tensor$<$*xf64$>$} --- безранговый тензор). Поэтому необходимо вывести формы для всех значений из известных форм.
Алгоритм вывода следующий:

\begin{enumerate}
	\item встроить (inline) все вызовы функций в main;
	\item распространить (propagate) формы через все вычисления:
	      \begin{enumerate}
		      \item \label{work_list}составить список всех операций которые возвращают безранговый тензор (это и будут те операции, для которых необходимо произвести вывод);
		      \item \label{loop} найти в списке \ref{work_list} операцию, у которой все операнды имеют тензор с определённой формой (если таковой не нашлось, то закончить вывод);
		      \item удалить операцию из списка \ref{work_list};
		      \item вывести форму для возвращаемого значения операции \ref{loop} из формы её операндов (например, если выводим для операции умножения (операция умножения поэлементная!), то форма возвращаемого значения равна форме любого операнда; для транспонирования форма возвращаемого значения равна перевёрнутой (reversed) форме операнда);
		      \item если список \ref{work_list} пустой, то завершить вывод, иначе перейти к \ref{loop}.
	      \end{enumerate}
\end{enumerate}

\subsubsection{Оптимизации для промежуточного представления, основанного на диалекте мини языка}

Хорошим примером иметь такое высокоуровневое промежуточное представление является возможность произвести оптимизацию, которая исключает последовательность двух транспонирований тензоров.

Для реализации этой оптимизаций было декларативно (при помощи TableGen) определено такое же правило перезаписи, как и на листинге \ref{rewrite_pat}.

Кроме того, были добавлены оптимизации для операции изменения формы тензора (<<toy.reshape>>):

\begin{itemize}
	\item если производилась попытка изменить форму тензора на ту же форму, то операция <<toy.reshape>> удалялась;
	\item при использовании операнда изменения формы тензора, форма изменялась непосредственно в операции <<toy.constant>>.
\end{itemize}

\textbf{Замечание}.
После этих оптимизаций операции <<toy.reshape>> больше нет в промежуточном представлении.

Более того, был добавлен проход (от англ. pass) <<cse>> --- com\-mon subexpression elimination (удаление общих подвыражений), который уже определён в MLIR.

\begin{lstlisting}[caption={Сгенерированное промежуточное представление для программы листинга \ref{toy_ex} после вывода форм и определения вышеупомянутых оптимизаций.}, frame=single]
module {
 toy.func @main() {
    %0 = toy.constant dense<[[1.0, 2.0, 3.0], [4.0, 5.0, 6.0]]>
                                            : tensor<2x3xf64>
    %1 = toy.transpose(%0 : tensor<2x3xf64>) to tensor<3x2xf64>
    %2 = toy.mul %1, %1 : tensor<3x2xf64>
    toy.print %2 : tensor<3x2xf64>
    toy.return
 }
}
\end{lstlisting}

\subsubsection{Понижение промежуточного представления в более низкоуровневое}

Для внедрения новых оптимизаций и дальнейшего понижения до LLVM IR необходимо понизить текущее высокоуровневое промежуточное представление в более низкоуровневое, которое в дальнейшем будем называть аффинным, состоящее из следующих диалектов:

\begin{itemize}
	\item affine\footnote{\url{https://mlir.llvm.org/docs/Dialects/Affine/} (дата доступа:   \DTMdate{2024-01-04}).} --- предоставляет абстракцию над аффинными преобразованиями. Для наших целей предоставляет операцию <<affine.\-for>>\footnote{\url{https://mlir.llvm.org/docs/Dialects/Affine/\#affinefor-affineaffineforop} (дата доступа:   \DTMdate{2024-01-04}).}, которая моделирует цикл;
	\item arith\footnote{\url{https://mlir.llvm.org/docs/Dialects/ArithOps/} (дата доступа:   \DTMdate{2024-01-04}).} --- предоставляет операции базовых математических операций;
	\item func\footnote{\url{https://mlir.llvm.org/docs/Dialects/Func/} (дата доступа:   \DTMdate{2024-01-04}).} диалект содержит операции, моделирующие создание и взаимодействие с функциями;
	\item memref\footnote{\url{https://mlir.llvm.org/docs/Dialects/MemRef/} (дата доступа:   \DTMdate{2024-01-04}).} --- предоставляет операции для управления памятью, которые используют тип \texttt{memref}\footnote{\url{https://mlir.llvm.org/docs/Dialects/Builtin/\#memreftype} (дата доступа:   \DTMdate{2024-01-04}).}, представляющий многомерный буфер.
\end{itemize}

Каждая операция из \ref{toy_ops} транслировалась в набор операций из вышеупомянутых диалектов.

\textbf{Замечание}.
После раннее добавленных проходов в промежуточном представлении больше нет операций <<toy.cast>>, <<toy.generic\_call>> и <<toy.reshape>>.

\begin{itemize}
	\item Для операции <<toy.constant>> сначала генерировались операции <<arith.\-cons\-tant>>, которые представляют обычные \texttt{float64} константы, затем аллоцировался и инициализировался этими константами буфер <<memref>>, который представлял в памяти тензор.
	\item Для операций <<toy.add>> и <<toy.mul>> генерировались вложенные циклы <<af\-fine.\-for>>, итерирующиеся по всем элементам буфера, в теле которых происходило поэлементное складывание или умножение элементов буферов при помощи операций <<arith.\-mulf>> и <<arith.\-addf>>.
	\item Для операции <<toy.transpose>> генерировался дополнительный буфер с перевёрнутой (reversed) формой, и создавался цикл, который заполнял новый буфер в соответствии с семантикой операции транспонирования.
	\item Операции <<toy.func>> и <<toy.return>> прямолинейно транслировались в операции <<func.func>> и <<func.return>> соответственно.
	\item Операция <<toy.print>> транслируется в вызов внешней функции <<printMemrefF64>>, которая содержится в библиотеке <<libmlir\_\-run\-ner\_utils>> (эта библиотека содержится в MLIR и представляет из себя небольшую runtime библиотеку для облегчения ввода/вывода). Для этого сначала генерируется forward-declaration функции <<printMemrefF64>>, затем, т. к эта функция принимает безранговый буфер, для операнда операции <<toy.print>> генерируется операция <<memref.cast>>, которая конвертирует буфер с рангом в безранговый, потом генерируется операция <<func.call>> для вызова функции <<printMemrefF64>>.
\end{itemize}

\subsubsection{Оптимизации аффиного промежуточного представления}
После понижения в аффинное промежуточное представление были добавлены уже реализованные в MLIR проходы: <<cse>> (удаление общих подвыражений), <<affine-loop-fusion>>\footnote{\url{https://mlir.llvm.org/docs/Passes/\#-affine-loop-fusion} (дата доступа:   \DTMdate{2024-01-04}).} (слияние циклов), <<affine-scalar-replacement>>\footnote{\url{https://mlir.llvm.org/docs/Passes/\#-affine-scalrep} (дата доступа:   \DTMdate{2024-01-04}).} (удаляет избыточные обращения к буферам).

\begin{lstlisting}[caption={Сгенерированное промежуточное представление для программы листинга \ref{toy_ex} после понижения до аффинного промежуточного представления и добавления вышеупомянутых оптимизаций.}, frame=single]
module {
  func.func @printMemrefF64(memref<*xf64>)
  func.func @main() {
   %cst = arith.constant 1.000000e+00 : f64
   %cst_0 = arith.constant 2.000000e+00 : f64
   %cst_1 = arith.constant 3.000000e+00 : f64
   %cst_2 = arith.constant 4.000000e+00 : f64
   %cst_3 = arith.constant 5.000000e+00 : f64
   %cst_4 = arith.constant 6.000000e+00 : f64
   %alloc = memref.alloc() : memref<3x2xf64>
   %alloc_5 = memref.alloc() : memref<2x3xf64>
   affine.store %cst_4, %alloc_5[1, 2] : memref<2x3xf64>
   affine.store %cst_3, %alloc_5[1, 1] : memref<2x3xf64>
   affine.store %cst_2, %alloc_5[1, 0] : memref<2x3xf64>
   affine.store %cst_1, %alloc_5[0, 2] : memref<2x3xf64>
   affine.store %cst_0, %alloc_5[0, 1] : memref<2x3xf64>
   affine.store %cst, %alloc_5[0, 0] : memref<2x3xf64>
   affine.for %arg0 = 0 to 3 {
     affine.for %arg1 = 0 to 2 {
       %0 = affine.load %alloc_5[%arg1, %arg0] : memref<2x3xf64>
       %1 = arith.mulf %0, %0 : f64
       affine.store %1, %alloc[%arg0, %arg1] : memref<3x2xf64>
     }
   }
   %cast = memref.cast %alloc : memref<3x2xf64> to memref<*xf64>
   call @printMemrefF64(%cast) : (memref<*xf64>) -> ()
   memref.dealloc %alloc_5 : memref<2x3xf64>
   memref.dealloc %alloc : memref<3x2xf64>
   return
  }
}
\end{lstlisting}

\subsubsection{Понижение промежуточного представления в диалект LLVM}

Диалект LLVM\footnote{\url{https://mlir.llvm.org/docs/Dialects/LLVM/} (дата доступа:   \DTMdate{2024-01-04}).} полностью соответствует LLVM IR. Для большинства диалектов, которые находятся в репозитории MLIR, понижение в диалект LLVM уже определено. Поэтому понизить аффинное промежуточное представление не представляло труда.

\subsubsection{Понижение в LLVM IR и компиляция при помощи JIT}

После всех проделанных манипуляций оставалось лишь передать MLIR-модуль JIT-компилятору, который неявно понизит LLVM диалект в LLVM IR и скомпилирует программу, а также не забыть слинковать библиотеку <<libmlir\_\-run\-ner\_utils>>, которая используется для печати тензоров.

\begin{lstlisting}[caption={Вывод для программы листинга \ref{toy_ex}.}, frame=single]
[[1,   16],
 [4,   25],
 [9,   36]]
\end{lstlisting}
