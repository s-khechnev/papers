\documentclass{beamer}
%Для защит онлайн лучше использовать разрешение 16x9
%\documentclass[aspectratio=169]{beamer}

\input{preamble.tex}

% То, что в квадратных скобках, отображается внизу по центру каждого слайда.
\title[MLIR OCaml]{Реализация привязок к MLIR для OCaml}

% То, что в квадратных скобках, отображается в левом нижнем углу.
\institute[СПбГУ]{}

% То, что в квадратных скобках, отображается в левом нижнем углу.
\author[Хечнев Семен]{Семен Евгеньевич Хечнев, группа 21.Б07-мм}

\begin{document}
{
\setbeamertemplate{footline}{}
% Лого университета или организации, отображается в шапке титульного листа
\begin{frame}
	\includegraphics[width=1.4cm]{pictures/SPbGU_Logo.png}
	\vspace{-35pt}
	\hspace{-10pt}
	\begin{center}
		\begin{tabular}{c}
			\scriptsize{Санкт-Петербургский государственный университет} \\
			\scriptsize{Кафедра системного программирования}
		\end{tabular}
		\titlepage
	\end{center}

	\btVFill

	{\scriptsize
		\textbf{Научный руководитель:} Д.С. Косарев, ассистент кафедры системного программирования \\
	}
	\begin{center}
		\vspace{5pt}
		\scriptsize{Санкт-Петербург\\
			2023}
	\end{center}

\end{frame}
}

\begin{frame}[fragile]
	\frametitle{Введение}
	\begin{itemize}
		\item Существует множество инструментов, реализованных на разных языках
		\item Чтобы использовать решения, написанные на других языках программирования, реализуют привязки (от англ. bindings)
	\end{itemize}

	\begin{itemize}
		\item LLVM\footnote{\url{https://llvm.org/} (дата доступа:   \DTMdate{2024-01-04}).} --- инфраструктура для разработки компиляторов
		\item MLIR\footnote{\url{https://mlir.llvm.org/} (дата доступа:   \DTMdate{2024-01-04}).} (Multi-Level Intermediate Representation) --- подпроект LLVM, инфраструктура для определения и внедрения промежуточных представлений
		\item MLIR реализован на C++, хочется использовать в OCaml
	\end{itemize}
\end{frame}

\begin{frame}
	\frametitle{Обзор --- MLIR (1)}
	\textbf{MLIR} предоставляет:
	\begin{itemize}
		\item Гибкое базовое промежуточное представление в SSA (Static Single Assignment) форме
		      \begin{itemize}
			      \item SSA $\Leftrightarrow$ каждой переменной значение присваивается ровно один раз
		      \end{itemize}
		\item Инструменты для расширения и анализа IR (Intermediate Representation)
		\item Готовые проходы (passes)
		\item Инструменты для тестирования и отладки
	\end{itemize}
\end{frame}

\begin{frame}[fragile]
	\frametitle{Обзор --- MLIR (2)}
	Базовые \textbf{объекты} в MLIR:

	\begin{itemize}
		\item Основная сущность --- операция
		\item Атрибуты --- compile-time константы, прикреплённые к операциям
		\item Типы
		\item Блоки --- список операций, заканчивающийся терминатором
		\item Регионы --- список блоков
	\end{itemize}

	\begin{lstlisting}[caption = {Пример операции}, frame=single]
^bb1:
   %res:2 = "mydialect.op"(%arg1) { attr = 3 }
      : (i32) -> (f32, i8) loc("mysrc.ml":10:8)
   ...
\end{lstlisting}

\end{frame}

\begin{frame}[fragile]
	\frametitle{Обзор --- MLIR (3)}
	\begin{itemize}
		\item Для поддержки расширяемости IR MLIR использует диалекты
		\item Диалект --- пространство имён для операций, типов и атрибутов
		\item Базовое промежуточное представление --- диалект <<builtin>>\footnote{\url{https://mlir.llvm.org/docs/Dialects/Builtin/} (дата доступа:   \DTMdate{2024-01-04}).}
	\end{itemize}
\end{frame}

% \begin{frame}[fragile]
% 	\frametitle{Обзор --- MLIR TableGen}
% 	\begin{itemize}
% 		\item 	Для определения новых диалектов, операций, правил переписывания IR MLIR предоставляет декларативный DSL --- TableGen
% 	\end{itemize}
% 	\begin{lstlisting}[caption={Определение диалекта <<MyDialect>> при помощи TableGen}, frame=single]
% def MyDialect : Dialect {
%     let name = "myDialect";

%     let summary = "My dialect.";
%     let description = [{ My cool dialect. }];

%     let cppNamespace = "myDialect";
% }
% \end{lstlisting}

% \end{frame}

\begin{frame}
	\frametitle{Постановка задачи}
	\textbf{Целью} работы является реализация привязок к MLIR для OCaml %озвученной выше

	\textbf{Задачи}:
	\begin{itemize}
		\item Реализовать привязки к MLIR
		\item Разработать компилятор для мини языка программирования с использованием разработанных привязок
		\item Реализовать тесты
	\end{itemize}
\end{frame}

\begin{frame}[fragile]
	\frametitle{Обзор --- ctypes}

	\begin{itemize}
		\item Ctypes\footnote{\url{https://github.com/yallop/ocaml-ctypes} (дата доступа:   \DTMdate{2024-01-04}).} --- это OCaml библиотека для создания привязок к \textsc{С}
		      \begin{itemize}
			      \item Для создания привязки необходимо определить имя и сигнатуру связываемой функции
		      \end{itemize}
	\end{itemize}

	\begin{lstlisting}[caption={Пример привязки <<print\_int>> для функции С, которая принимает \texttt{int} и возвращает \texttt{void}},language=Caml, frame=single]
let print_int = foreign "print_int"
                           (int @-> returning void)
\end{lstlisting}

\end{frame}

\begin{frame}[fragile]
	\frametitle{Обзор --- существующие решения}
	\begin{itemize}
		\item За основу взяты существующие привязки\footnote{\url{https://github.com/tachukao/ocaml-mlir} (дата доступа:   \DTMdate{2024-01-04}).} к MLIR
		\item Особенности:
		      \begin{itemize}
			      \item Реализованы к форку MLIR 11 версии
			      \item Не примитивного примера использования нет
			      \item Больше не поддерживаются
		      \end{itemize}
	\end{itemize}
\end{frame}

\begin{frame}[fragile]
	\frametitle{Реализация --- привязки}
	\textbf{Что было сделано}:
	\begin{itemize}
		\item Выбрана целевая версия MLIR --- 16.0.6
		\item Обновлены зависимости и удалён устаревший код
		\item Добавлена возможность собирать привязки, используя dev пакеты\footnote{\url{https://apt.llvm.org/} (дата доступа:   \DTMdate{2024-01-04}).} MLIR и LLVM
		\item Реализовано большинство недостающих привязок
	\end{itemize}
\end{frame}

\begin{frame}[fragile]
	\frametitle{Реализация --- мини язык программирования (1)}

	\begin{itemize}
		\item Тип данных --- тензоры (ранга 0--2) с элементами типа \texttt{float64}
		\item Операции:
		      \begin{itemize}
			      \item печать
			      \item поэлементное сложение и умножение
			      \item транспонирование
			      \item изменение формы тензора
		      \end{itemize}
		\item Можно определять функции
		\item Переменные неизменяемы, статическая типизация, вывод типов
	\end{itemize}

	\begin{lstlisting}[caption={Пример программы на мини языке программирования}, frame=single]
def main() {
    var a = [[1, 2, 3], [4, 5, 6]];
    var b<3, 2> = [1, 2, 3, 4, 5, 6];
    print(a * transpose(b));
}
\end{lstlisting}

	\blfootnote{\url{https://mlir.llvm.org/docs/Tutorials/Toy/}}

\end{frame}

\begin{frame}[fragile]
	\frametitle{Реализация --- мини язык программирования (2)}
	\begin{itemize}
		\item Процесс компиляции:
		      \begin{itemize}
			      \item Парсинг в AST
			      \item Генерация из AST IR высокого уровня
			      \item Вывод типов
			      \item Оптимизация IR
			      \item Понижение (lowering) IR высокого уровня в более низкоуровневое
			      \item Оптимизация IR
			      \item Понижение в диалект LLVM
			      \item Понижение в LLVM IR
			      \item Компиляция LLVM IR при помощи JIT
		      \end{itemize}
	\end{itemize}
\end{frame}

\begin{frame}
	\frametitle{Тестирование}
	\begin{itemize}
		\item Для привязок уже были реализованы тесты, оставалось внести небольшие правки, чтобы они работали для версии 16.0.6
		\item Для каждого этапа компиляции мини языка программирования были реализованы тесты
		\item CI
		      \begin{itemize}
			      \item Сборка
			      \item Запуск тестов
			      \item Обновление отчёта о покрытии
		      \end{itemize}

		\item Тестовое покрытие
		      \begin{itemize}
			      \item Покрытие всего проекта --- 55\%
			            \begin{itemize}
				            \item Привязок --- 46\%
				            \item Компилятора мини языка --- 96\%
			            \end{itemize}
		      \end{itemize}
	\end{itemize}

\end{frame}

\begin{frame}
	\frametitle{Заключение}
	\textbf{Результаты}:
	\begin{itemize}
		\item Реализованы привязки к MLIR
		\item Разработан компиляторо для мини языка программирования с использованием реализованных привязок
		\item Реализованы тесты
	\end{itemize}

	\textbf{Планы на будущее}:
	\begin{itemize}
		\item Реализовать компилятор для подмножества некоторого ML языка с использованием MLIR
	\end{itemize}

	\blfootnote{Код проекта: \url{https://github.com/s-khechnev/ocaml-mlir}}
	\blfootnote{Имя аккаунта: s-khechnev}

\end{frame}

\end{document}
