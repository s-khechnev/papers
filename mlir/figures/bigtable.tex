% !TeX spellcheck = ru_RU
% !TEX root = vkr.tex

\newcolumntype{C}{ >{\centering\arraybackslash} m{4cm} }
\newcommand\myvert[1]{\rotatebox[origin=c]{90}{#1}}
\newcommand\myvertcell[1]{\multirowcell{5}{\myvert{#1}}}
\newcommand\myvertcelll[1]{\multirowcell{4}{\myvert{#1}}}
\newcommand\myvertcellN[2]{\multirowcell{#1}{\myvert{#2}}}


\afterpage{%
    \clearpage% Flush earlier floats (otherwise order might not be correct)
    \thispagestyle{empty}% empty page style (?)
    \begin{landscape}% Landscape page
        \centering % Center table

\begin{tabular}{|c|c|c|c|c|c|c|c|c|c|c|c|c|c|c|c|c|c|}\hline
%& \multicolumn{17}{c|}{} \\ \hline
\multirowcell{2}{Код модуля \\в составе \\ дисциплины,\\практики и т.п. }
  &\myvertcellN{2}{Трудоёмкость\quad}
  & \multicolumn{10}{c|}{\tiny{Контактная работа обучающихся с преподавателем}}
  & \multicolumn{5}{c|}{\tiny{Самостоятельная работа}}
  & \myvertcellN{2}{\tiny Объем активных и интерактивных\quad}
  \\ \cline{3-17}

&& \myvertcellN{2}{лекции\quad}
    &\myvertcellN{2}{семинары\quad}
    &\myvertcellN{2}{консультации\quad}
    &\myvertcellN{2}{\small практические  занятия\quad}
    &\myvertcellN{2}{\small лабораторные работы\quad}
    &\myvertcellN{2}{\small контрольные работы\quad}
    &\myvertcellN{2}{\small коллоквиумы\quad}
    &\myvertcellN{2}{\small текущий контроль\quad}
    &\myvertcellN{2}{\small промежуточная аттестация\quad}
    &\myvertcellN{2}{\small итоговая аттестация\quad}

    &\myvertcellN{2}{\tiny под руководством    преподавателя\quad}
    &\myvertcellN{2}{\tiny в присутствии     преподавателя\quad}
    &\myvertcellN{2}{\tiny с использованием    методических\quad}
    &\myvertcellN{2}{\small текущий контроль\quad}
    &\myvertcellN{2}{\makecell{\small промежуточная \\ аттестация}}
    &     \\
&& &&&&&&&&& &&&&&&\\
&& &&&&&&&&& &&&&&&\\
&& &&&&&&&&& &&&&&&\\
&&&&&&&&&&& &&&&&&\\
&&&&&&&&&&& &&&&&&\\
&&&&&&&&&&& &&&&&&\\ \hline
Семестр 3 & 2 &30  &&&&&&&&2   & &&&18 &&20 &10\\ \hline
          &   &2--42&&&&&&&&2--25& &&&1--1&&1--1&\\ \hline
Итого     & 2 &30  &&&&&&&&2   & &&&18 &&20 &10\\ \hline
\end{tabular}

        \captionof{table}{Если таблица очень большая, то можно её изобразить на отдельной портретной странице. Не забудьте подробное описание, чтобы содержимое таблицы можно было понять не читая весь текст.}
    \end{landscape}
    \clearpage% Flush page
}

